\subsection{Relevant Work}

The application of computer vision and machine learning to sports video analytics has seen a great development over the past few years. Systems have been developed and studies researched in various domains to automate player detection, tracking, team classification, and spatial analysis. These solutions have aimed to replace traditional manual annotations with scalable, real-time pipelines that enable deeper tactical analysis. These studies have served as a stepping stone and showcased great advancements.

In 2013, Lu et al. proposed one of the first complete pipelines for player tracking from \textit{single pan-tilt-zoom camera} sports broadcast videos \cite{lu2013learning}. The system used a \textit{tracking-by-detection} approach where for the detection of players, a \textit{deformable part model (DPM)} was used, and a bipartite matching for player tracking. This tracking algorithm was based on the euclidean distances between centres of the detections and the predictive locations of tracks. This work also incorporated a player identification method which generated impressive results. It extracted and combined several visual features from the entire body of players and combined them into a Conditional Random Field (CRF). 


Another notable piece of work was produced by Spearman in 2018 \cite{spearman2018beyond}. The work introduced probabilistic physics-based models for football, quantifying the likelihood of players to score at any point during the match and where on the pitch their scoring is likely to come from. His work emphasized the importance of spatial dominance as a critical tactical feature, extending the usage of player tracking data beyond the basic positional statistics.

In the sport of ice hockey, Kutylbek developed a comprehensive player tracking system that combined \textit{YOLO}-based object detection with \textit{DeepSORT} for tracking, \textit{HSV} histogram-based team classification, and homography transformation to map player positions onto a standardized 2D rink layout. The system achieved a high accuracy, with a mean Average Precision (mAP) of 96.8\% for player detection, and facilitated detailed tactical evaluations such as spatial control and player heatmaps \cite{kutylbek2024tracking}.

Similar techniques have been adapted for football in the past. Patel constructed a system focused on tactical football footage. The detection and tracking was similar to Kutylbek's work, whereas team classification was handled using \textit{MeanShift} clustering applied to RGB colour histograms. Furthermore, a homography transformation was employed to generate bird's-eye view mappings of the pitch, enabling the generation of heatmaps and spatial distribution plots for performance analysis. His system demonstrated excellent tracking accuracy, achieving a player Multiple Object Tracking Accuracy (MOTA) of 99.63\% \cite{patel2024tracking} .

As football and ice hockey saw developments, volleyball was not left behind. Paulinus designed a system for player and court tracking, using \textit{YOLOv8} for both tasks. Optical flow techniques were further applied to track the movement of the volleyball court when the camera moved dynamically. His work emphasized not only detection and tracking, but also dynamic top-down transformation of player positions, thus enabling visualizations of team structures and positional patterns during matches \cite{paulinus2024tracking}.


On the commercial side of things, systems such as \textit{TRACAB}~\cite{tracab} and \textit{Genius Sports}~\cite{genius} have been dominant in professional football analytics. \textit{TRACAB} utilizes the multiple synchronized cameras installed across stadiums. Using the data obtained and coupling it with deep-learning based object tracking, they deliver player coordinates at a high frequency. Genius Sports on the other hand, deploys computer vision and AI pipelines to deliver a real-time tracking along with tactical visualizations as well as predictive analytics. These systems, however, rely on multi-camera setups and proprietary models, making them less accessible in comparison to the open research approaches.


Overall, the literature reflects a common system architecture across sports: object detection for player localization, tracking for the preservation of identity, team classification based on the visual features, spatial mapping through homography, and a tactical analysis through various spatial metrics such as Voronoi diagrams or pitch control probabilities. The project builds on this pipeline, specifically targeting football, while integrating modern techniques such as \textit{SigLIP-based embeddings} for more robust team classification and advanced spatial visualizations like Voronoi diagrams and offside line detections.

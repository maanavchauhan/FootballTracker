\subsection{Motivation}




The game of \textit{football} is one of the most globally celebrated and commercially significant sports, capturing the attention of millions of fans and analysts worldwide. With the growth of advanced analytics seen in the recent years, there has been a marked shift in how teams, broadcasters, and fans engage with the game. Football Clubs have started to rely deeply on data-driven insights for overall performance evaluation, tactical analysis, and even individual player development. Similarly, broadcasters and sports technology firms are seeking more automated and intelligent ways of delivering richer, real-time insights to their audiences. At the core of many of these efforts lies the ability to accurately detect, track, and analyse the movements of players and the ball throughout a match.

The motivation for this project stems from the challenge and opportunity of applying computer vision and deep learning techniques to this fast-paced, dynamic environment. Football poses multiple unique problems for visual tracking: players often do occlude one another, lighting and camera angles can vary severely, and the ball itself is small and moves unpredictably and at a fast pace. Despite these challenges, solving this problem has the potential to unlock high-impact use cases; from automatic event detection and tactical heatmaps to advanced player metrics and broadcast enhancements.

This project seeks to develop a robust system that leverages object detection and tracking algorithms (such as \textit{YOLO} and \textit{ByteTrack}) to identify and follow players and the ball in football footage. By focusing on football, the project not only contributes to the technical field of computer vision but also addresses a real-world need in sports technology. The use of machine learning in football analytics is still an evolving area, and there is a clear demand for systems that can operate efficiently, accurately, and with minimal human intervention.

In pursuing this project, the goal is not only to implement existing methods but also to understand their limitations, explore ways to improve tracking reliability, and make contributions that could eventually scale to professional use. Ultimately, this work is driven by both, a technical curiosity about complex visual systems and a deep passion for football, a sport where every movement counts.

